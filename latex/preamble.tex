% THis file contains all the default packages and modifications for
% LHCb formatting

%% %%%%%%%%%%%%%%%%%%
%%  Page formatting
%% %%%%%%%%%%%%%%%%%%
%%\usepackage[margin=1in]{geometry}
\usepackage[top=1in, bottom=1.25in, left=1in, right=1in]{geometry}
\usepackage[utf8]{inputenc}

% fallback for manual settings... uncomment if the geometry package is not available
%
%\voffset=-11mm
%\textheight=220mm
%\textwidth=160mm
%\oddsidemargin=0mm
%\evensidemargin=0mm

\columnsep=5mm
\addtolength{\belowcaptionskip}{0.5em}

\renewcommand{\textfraction}{0.01}
\renewcommand{\floatpagefraction}{0.99}
\renewcommand{\topfraction}{0.9}
\renewcommand{\bottomfraction}{0.9}

% Allow the page size to vary a bit ...
\raggedbottom
% To avoid Latex to be too fussy with line breaking ...
\sloppy

%% %%%%%%%%%%%%%%%%%%%%%%%
%% Packages to be used
%% %%%%%%%%%%%%%%%%%%%%%%%
\usepackage{microtype}
\usepackage{lineno}  % for line numbering during review
\usepackage{xspace} % To avoid problems with missing or double spaces after
                    % predefined symbold
\usepackage{caption} %these three command get the figure and table captions automatically small
\renewcommand{\captionfont}{\small}
\renewcommand{\captionlabelfont}{\small}

%% Graphics
\usepackage{graphicx}  % to include figures (can also use other packages)
\usepackage{color}
\usepackage{colortbl}
\usepackage{booktabs}
\usepackage{multirow}
\usepackage{rotating} % define environment sidewaystable to rotate tables
\usepackage{placeins} %Define \FloatBarrier

\graphicspath{{./figures/},{./generated/}} % Make Latex search fig subdir for figures, should
                                           % match FIGDIR and GENDIR values in Makefile


%% Math
\usepackage{xfrac} % Adds \sfrac
\usepackage{amsmath} % Adds a large collection of math symbols
\usepackage{amssymb}
\usepackage{amsfonts}
\usepackage{upgreek} % Adds in support for greek letters in roman typeset

%% fix to allow peaceful coexistence of line numbering and
%% mathematical objects
%% http://www.latex-community.org/forum/viewtopic.php?f=5&t=163
%%
\newcommand*\patchAmsMathEnvironmentForLineno[1]{%
\expandafter\let\csname old#1\expandafter\endcsname\csname #1\endcsname
\expandafter\let\csname oldend#1\expandafter\endcsname\csname
end#1\endcsname
 \renewenvironment{#1}%
   {\linenomath\csname old#1\endcsname}%
   {\csname oldend#1\endcsname\endlinenomath}%
}
\newcommand*\patchBothAmsMathEnvironmentsForLineno[1]{%
  \patchAmsMathEnvironmentForLineno{#1}%
  \patchAmsMathEnvironmentForLineno{#1*}%
}
\AtBeginDocument{%
\patchBothAmsMathEnvironmentsForLineno{equation}%
\patchBothAmsMathEnvironmentsForLineno{align}%
\patchBothAmsMathEnvironmentsForLineno{flalign}%
\patchBothAmsMathEnvironmentsForLineno{alignat}%
\patchBothAmsMathEnvironmentsForLineno{gather}%
\patchBothAmsMathEnvironmentsForLineno{multline}%
\patchBothAmsMathEnvironmentsForLineno{eqnarray}%
}

\usepackage{siunitx}
\sisetup{
  % Use '10^{n}' notation
  scientific-notation=true,
  % Display uncertainties as '1.23 \pm 0.05' rather than '1.23(5)'
  separate-uncertainty=true,
  % Separate thousands with a comma, with numbers with more than four digits
  group-separator = {,},
  % Separate ranges with an en-dash
  range-phrase=--,
  % Only give the unit on the last value
  range-units=single
}

% Get hyperlinks to captions and in references.
% These do not work with revtex. Use "hypertext" as class option instead.
\usepackage{hyperxmp}
\usepackage[pdftex,
            pdfauthor={\paperauthors},
            pdftitle={\paperasciititle},
            pdfsubject={\papersubtitle},
            pdfkeywords={\paperkeywords},
            pdfcopyright={Copyright (C) \papercopyright},
            pdflicenseurl={\paperlicenceurl}]{hyperref}
\usepackage[all]{hypcap} % Internal hyperlinks to floats.
\usepackage[backend=biber,style=phys,eprint=true,hyperref=true,biblabel=brackets]{biblatex}
\DefineBibliographyStrings{english}{%
  andothers = {\etal} % use italic et. al.
}
\DeclareFieldFormat*{title}{\textit{#1}} % put paper titles in italics
\DeclareSourcemap{
  \maps[datatype=bibtex,overwrite=true]{
    \map{
      \step[fieldset=eprintclass,fieldvalue=] % don't show e.g. [hep-ex] in the bibliography
    }
    \map{
      \step[fieldsource=collaboration,final=true] % only do anything if the collaboration is set
      \step[fieldset=usera,origfieldval,final=true]
    }
    \map{
      \step[fieldsource=note,final=true] % only do anything if there's a note
      \step[fieldset=addendum,origfieldval] % make the note an addendum
      \step[fieldset=note,fieldvalue=] % remove the note
    }
  }
}
\renewbibmacro*{author}{%
  \iffieldundef{usera}{%
    \printnames{author} % A. Author et. al.
  }{%
    \printfield{usera}, \printnames{author} % COLLABORATION, A. Author et. al.
%    \printfield{usera} % COLLABORATION
  }
}
\addbibresource{bibliography/refs.bib}
\addbibresource{bibliography/main.bib}
\addbibresource{bibliography/extras.bib}
\addbibresource{bibliography/LHCb-DP.bib}
\addbibresource{bibliography/LHCb-TDR.bib}
\addbibresource{bibliography/LHCb-CONF.bib}
\addbibresource{bibliography/LHCb-PAPER.bib}

%%% $Id$
%%% ======================================================================
%%% Purpose: Standard LHCb aliases
%%% Author: Originally Ulrik Egede, adapted by Tomasz Skwarnicki for templates,
%%% rewritten by Chris Parkes
%%% Maintainer : Ulrik Egede (2010 - 2012)
%%% Maintainer : Rolf Oldeman (2012 - 2014)
%%% =======================================================================

%%% To use this file outside the normal LHCb document environment, the
%%% following should be added in a preamble (before \begin{document}
%%%
%%%\usepackage{ifthen} 
%%%\newboolean{uprightparticles}
%%%\setboolean{uprightparticles}{false} %Set true for upright particle symbols
\usepackage{xspace} 
\usepackage{upgreek}

%%%%%%%%%%%%%%%%%%%%%%%%%%%%%%%%%%%%%%%%%%%%%%%%%%%%%%%%%%%%
%%%
%%% The following is to ensure that the template automatically can process
%%% this file.
%%%
%%% Add comments with at least three %%% preceding.
%%% Add new sections with one % preceding
%%% Add new subsections with two %% preceding
%%%%%%%%%%%%%%%%%%%%%%%%%%%%%%%%%%%%%%%%%%%%%%%%%%%%%%%%%%%%

%%%%%%%%%%%%%
% Experiments
%%%%%%%%%%%%%
\def\lhcb {\mbox{LHCb}\xspace}
\def\atlas  {\mbox{ATLAS}\xspace}
\def\cms    {\mbox{CMS}\xspace}
\def\alice  {\mbox{ALICE}\xspace}
\def\babar  {\mbox{BaBar}\xspace}
\def\belle  {\mbox{Belle}\xspace}
\def\cleo   {\mbox{CLEO}\xspace}
\def\cdf    {\mbox{CDF}\xspace}
\def\dzero  {\mbox{D\O}\xspace}
\def\aleph  {\mbox{ALEPH}\xspace}
\def\delphi {\mbox{DELPHI}\xspace}
\def\opal   {\mbox{OPAL}\xspace}
\def\lthree {\mbox{L3}\xspace}
\def\sld    {\mbox{SLD}\xspace}
%%%\def\argus  {\mbox{ARGUS}\xspace}
%%%\def\uaone  {\mbox{UA1}\xspace}
%%%\def\uatwo  {\mbox{UA2}\xspace}
%%%\def\ux85 {\mbox{UX85}\xspace}
\def\cern {\mbox{CERN}\xspace}
\def\lhc    {\mbox{LHC}\xspace}
\def\lep    {\mbox{LEP}\xspace}
\def\tevatron {Tevatron\xspace}

%% LHCb sub-detectors and sub-systems

%%%\def\pu     {PU\xspace}
\def\velo   {VELO\xspace}
\def\rich   {RICH\xspace}
\def\richone {RICH1\xspace}
\def\richtwo {RICH2\xspace}
\def\ttracker {TT\xspace}
\def\intr   {IT\xspace}
\def\st     {ST\xspace}
\def\ot     {OT\xspace}
\def\herschel {\mbox{\textsc{HeRSCheL}}\xspace}
%%%\def\Tone   {T1\xspace}
%%%\def\Ttwo   {T2\xspace}
%%%\def\Tthree {T3\xspace}
%%%\def\Mone   {M1\xspace}
%%%\def\Mtwo   {M2\xspace}
%%%\def\Mthree {M3\xspace}
%%%\def\Mfour  {M4\xspace}
%%%\def\Mfive  {M5\xspace}
\def\spd    {SPD\xspace}
\def\presh  {PS\xspace}
\def\ecal   {ECAL\xspace}
\def\hcal   {HCAL\xspace}
%%%\def\bcm    {BCM\xspace}
\def\MagUp {\mbox{\em Mag\kern -0.05em Up}\xspace}
\def\MagDown {\mbox{\em MagDown}\xspace}

\def\ode    {ODE\xspace}
\def\daq    {DAQ\xspace}
\def\tfc    {TFC\xspace}
\def\ecs    {ECS\xspace}
\def\lone   {L0\xspace}
\def\hlt    {HLT\xspace}
\def\hltone {HLT1\xspace}
\def\hlttwo {HLT2\xspace}

%%% Redefine these with per-particle conditionals so that you can switch partway through the document (e.g. upright for
%%% the main body, slanted for the bibliography
%%% DeclareRobustCommand to avoid problems using these inside figure captions etc.
\DeclareRobustCommand{\Pa}{\ensuremath{\ifthenelse{\boolean{uprightparticles}}{\mathrm{a}}{a}}\xspace}
\DeclareRobustCommand{\Pb}{\ensuremath{\ifthenelse{\boolean{uprightparticles}}{\mathrm{b}}{b}}\xspace}
\DeclareRobustCommand{\Pc}{\ensuremath{\ifthenelse{\boolean{uprightparticles}}{\mathrm{c}}{c}}\xspace}
\DeclareRobustCommand{\Pd}{\ensuremath{\ifthenelse{\boolean{uprightparticles}}{\mathrm{d}}{d}}\xspace}
\DeclareRobustCommand{\Pe}{\ensuremath{\ifthenelse{\boolean{uprightparticles}}{\mathrm{e}}{e}}\xspace}
\DeclareRobustCommand{\Pf}{\ensuremath{\ifthenelse{\boolean{uprightparticles}}{\mathrm{f}}{f}}\xspace}
\DeclareRobustCommand{\Pg}{\ensuremath{\ifthenelse{\boolean{uprightparticles}}{\mathrm{g}}{g}}\xspace}
\DeclareRobustCommand{\Ph}{\ensuremath{\ifthenelse{\boolean{uprightparticles}}{\mathrm{h}}{h}}\xspace}
\DeclareRobustCommand{\Pi}{\ensuremath{\ifthenelse{\boolean{uprightparticles}}{\mathrm{i}}{i}}\xspace}
\DeclareRobustCommand{\Pj}{\ensuremath{\ifthenelse{\boolean{uprightparticles}}{\mathrm{j}}{j}}\xspace}
\DeclareRobustCommand{\Pk}{\ensuremath{\ifthenelse{\boolean{uprightparticles}}{\mathrm{k}}{k}}\xspace}
\DeclareRobustCommand{\Pl}{\ensuremath{\ifthenelse{\boolean{uprightparticles}}{\mathrm{l}}{l}}\xspace}
\DeclareRobustCommand{\Pm}{\ensuremath{\ifthenelse{\boolean{uprightparticles}}{\mathrm{m}}{m}}\xspace}
\DeclareRobustCommand{\Pn}{\ensuremath{\ifthenelse{\boolean{uprightparticles}}{\mathrm{n}}{n}}\xspace}
\DeclareRobustCommand{\Po}{\ensuremath{\ifthenelse{\boolean{uprightparticles}}{\mathrm{o}}{o}}\xspace}
\DeclareRobustCommand{\Pp}{\ensuremath{\ifthenelse{\boolean{uprightparticles}}{\mathrm{p}}{p}}\xspace}
\DeclareRobustCommand{\Pq}{\ensuremath{\ifthenelse{\boolean{uprightparticles}}{\mathrm{q}}{q}}\xspace}
\DeclareRobustCommand{\Pr}{\ensuremath{\ifthenelse{\boolean{uprightparticles}}{\mathrm{r}}{r}}\xspace}
\DeclareRobustCommand{\Ps}{\ensuremath{\ifthenelse{\boolean{uprightparticles}}{\mathrm{s}}{s}}\xspace}
\DeclareRobustCommand{\Pt}{\ensuremath{\ifthenelse{\boolean{uprightparticles}}{\mathrm{t}}{t}}\xspace}
\DeclareRobustCommand{\Pu}{\ensuremath{\ifthenelse{\boolean{uprightparticles}}{\mathrm{u}}{u}}\xspace}
\DeclareRobustCommand{\Pv}{\ensuremath{\ifthenelse{\boolean{uprightparticles}}{\mathrm{v}}{v}}\xspace}
\DeclareRobustCommand{\Pw}{\ensuremath{\ifthenelse{\boolean{uprightparticles}}{\mathrm{w}}{w}}\xspace}
\DeclareRobustCommand{\Px}{\ensuremath{\ifthenelse{\boolean{uprightparticles}}{\mathrm{x}}{x}}\xspace}
\DeclareRobustCommand{\Py}{\ensuremath{\ifthenelse{\boolean{uprightparticles}}{\mathrm{y}}{y}}\xspace}
\DeclareRobustCommand{\Pz}{\ensuremath{\ifthenelse{\boolean{uprightparticles}}{\mathrm{z}}{z}}\xspace}
\DeclareRobustCommand{\PA}{\ensuremath{\ifthenelse{\boolean{uprightparticles}}{\mathrm{A}}{A}}\xspace}
\DeclareRobustCommand{\PB}{\ensuremath{\ifthenelse{\boolean{uprightparticles}}{\mathrm{B}}{B}}\xspace}
\DeclareRobustCommand{\PC}{\ensuremath{\ifthenelse{\boolean{uprightparticles}}{\mathrm{C}}{C}}\xspace}
\DeclareRobustCommand{\PD}{\ensuremath{\ifthenelse{\boolean{uprightparticles}}{\mathrm{D}}{D}}\xspace}
\DeclareRobustCommand{\PE}{\ensuremath{\ifthenelse{\boolean{uprightparticles}}{\mathrm{E}}{E}}\xspace}
\DeclareRobustCommand{\PF}{\ensuremath{\ifthenelse{\boolean{uprightparticles}}{\mathrm{F}}{F}}\xspace}
\DeclareRobustCommand{\PG}{\ensuremath{\ifthenelse{\boolean{uprightparticles}}{\mathrm{G}}{G}}\xspace}
\DeclareRobustCommand{\PH}{\ensuremath{\ifthenelse{\boolean{uprightparticles}}{\mathrm{H}}{H}}\xspace}
\DeclareRobustCommand{\PI}{\ensuremath{\ifthenelse{\boolean{uprightparticles}}{\mathrm{I}}{I}}\xspace}
\DeclareRobustCommand{\PJ}{\ensuremath{\ifthenelse{\boolean{uprightparticles}}{\mathrm{J}}{J}}\xspace}
\DeclareRobustCommand{\PK}{\ensuremath{\ifthenelse{\boolean{uprightparticles}}{\mathrm{K}}{K}}\xspace}
\DeclareRobustCommand{\PL}{\ensuremath{\ifthenelse{\boolean{uprightparticles}}{\mathrm{L}}{L}}\xspace}
\DeclareRobustCommand{\PM}{\ensuremath{\ifthenelse{\boolean{uprightparticles}}{\mathrm{M}}{M}}\xspace}
\DeclareRobustCommand{\PN}{\ensuremath{\ifthenelse{\boolean{uprightparticles}}{\mathrm{N}}{N}}\xspace}
\DeclareRobustCommand{\PO}{\ensuremath{\ifthenelse{\boolean{uprightparticles}}{\mathrm{O}}{O}}\xspace}
\DeclareRobustCommand{\PP}{\ensuremath{\ifthenelse{\boolean{uprightparticles}}{\mathrm{P}}{P}}\xspace}
\DeclareRobustCommand{\PQ}{\ensuremath{\ifthenelse{\boolean{uprightparticles}}{\mathrm{Q}}{Q}}\xspace}
\DeclareRobustCommand{\PR}{\ensuremath{\ifthenelse{\boolean{uprightparticles}}{\mathrm{R}}{R}}\xspace}
\DeclareRobustCommand{\PS}{\ensuremath{\ifthenelse{\boolean{uprightparticles}}{\mathrm{S}}{S}}\xspace}
\DeclareRobustCommand{\PT}{\ensuremath{\ifthenelse{\boolean{uprightparticles}}{\mathrm{T}}{T}}\xspace}
\DeclareRobustCommand{\PU}{\ensuremath{\ifthenelse{\boolean{uprightparticles}}{\mathrm{U}}{U}}\xspace}
\DeclareRobustCommand{\PV}{\ensuremath{\ifthenelse{\boolean{uprightparticles}}{\mathrm{V}}{V}}\xspace}
\DeclareRobustCommand{\PW}{\ensuremath{\ifthenelse{\boolean{uprightparticles}}{\mathrm{W}}{W}}\xspace}
\DeclareRobustCommand{\PX}{\ensuremath{\ifthenelse{\boolean{uprightparticles}}{\mathrm{X}}{X}}\xspace}
\DeclareRobustCommand{\PY}{\ensuremath{\ifthenelse{\boolean{uprightparticles}}{\mathrm{Y}}{Y}}\xspace}
\DeclareRobustCommand{\PZ}{\ensuremath{\ifthenelse{\boolean{uprightparticles}}{\mathrm{Z}}{Z}}\xspace}
\DeclareRobustCommand{\PDelta}{\ensuremath{\ifthenelse{\boolean{uprightparticles}}{\Delta}{\mathchar"7101}}\xspace}
\DeclareRobustCommand{\PLambda}{\ensuremath{\ifthenelse{\boolean{uprightparticles}}{\Lambda}{\mathchar"7103}}\xspace}
\DeclareRobustCommand{\POmega}{\ensuremath{\ifthenelse{\boolean{uprightparticles}}{\Omega}{\mathchar"710A}}\xspace}
\DeclareRobustCommand{\PSigma}{\ensuremath{\ifthenelse{\boolean{uprightparticles}}{\Sigma}{\mathchar"7106}}\xspace}
\DeclareRobustCommand{\PUpsilon}{\ensuremath{\ifthenelse{\boolean{uprightparticles}}{\Upsilon}{\mathchar"7107}}\xspace}
\DeclareRobustCommand{\PXi}{\ensuremath{\ifthenelse{\boolean{uprightparticles}}{\Xi}{\mathchar"7104}}\xspace}
\DeclareRobustCommand{\Palpha}{\ensuremath{\ifthenelse{\boolean{uprightparticles}}{\upalpha}{\alpha}}\xspace}
\DeclareRobustCommand{\Pbeta}{\ensuremath{\ifthenelse{\boolean{uprightparticles}}{\upbeta}{\beta}}\xspace}
\DeclareRobustCommand{\Pchi}{\ensuremath{\ifthenelse{\boolean{uprightparticles}}{\upchi}{\chi}}\xspace}
\DeclareRobustCommand{\Pdelta}{\ensuremath{\ifthenelse{\boolean{uprightparticles}}{\updelta}{\delta}}\xspace}
\DeclareRobustCommand{\Pepsilon}{\ensuremath{\ifthenelse{\boolean{uprightparticles}}{\upepsilon}{\epsilon}}\xspace}
\DeclareRobustCommand{\Peta}{\ensuremath{\ifthenelse{\boolean{uprightparticles}}{\upeta}{\eta}}\xspace}
\DeclareRobustCommand{\Pgamma}{\ensuremath{\ifthenelse{\boolean{uprightparticles}}{\upgamma}{\gamma}}\xspace}
\DeclareRobustCommand{\Piota}{\ensuremath{\ifthenelse{\boolean{uprightparticles}}{\upiota}{\iota}}\xspace}
\DeclareRobustCommand{\Pkappa}{\ensuremath{\ifthenelse{\boolean{uprightparticles}}{\upkappa}{\kappa}}\xspace}
\DeclareRobustCommand{\Plambda}{\ensuremath{\ifthenelse{\boolean{uprightparticles}}{\uplambda}{\lambda}}\xspace}
\DeclareRobustCommand{\Pmu}{\ensuremath{\ifthenelse{\boolean{uprightparticles}}{\upmu}{\mu}}\xspace}
\DeclareRobustCommand{\Pnu}{\ensuremath{\ifthenelse{\boolean{uprightparticles}}{\upnu}{\nu}}\xspace}
\DeclareRobustCommand{\Pomega}{\ensuremath{\ifthenelse{\boolean{uprightparticles}}{\upomega}{\omega}}\xspace}
\DeclareRobustCommand{\Pphi}{\ensuremath{\ifthenelse{\boolean{uprightparticles}}{\upphi}{\phi}}\xspace}
\DeclareRobustCommand{\Ppi}{\ensuremath{\ifthenelse{\boolean{uprightparticles}}{\uppi}{\pi}}\xspace}
\DeclareRobustCommand{\Ppsi}{\ensuremath{\ifthenelse{\boolean{uprightparticles}}{\uppsi}{\psi}}\xspace}
\DeclareRobustCommand{\Prho}{\ensuremath{\ifthenelse{\boolean{uprightparticles}}{\uprho}{\rho}}\xspace}
\DeclareRobustCommand{\Ptau}{\ensuremath{\ifthenelse{\boolean{uprightparticles}}{\uptau}{\tau}}\xspace}
\DeclareRobustCommand{\Ptheta}{\ensuremath{\ifthenelse{\boolean{uprightparticles}}{\uptheta}{\theta}}\xspace}
\DeclareRobustCommand{\Pupsilon}{\ensuremath{\ifthenelse{\boolean{uprightparticles}}{\upupsilon}{\upsilon}}\xspace}
\DeclareRobustCommand{\Pvarepsilon}{\ensuremath{\ifthenelse{\boolean{uprightparticles}}{\upvarepsilon}{\varepsilon}}\xspace}
\DeclareRobustCommand{\Pvarphi}{\ensuremath{\ifthenelse{\boolean{uprightparticles}}{\upvarphi}{\varphi}}\xspace}
\DeclareRobustCommand{\Pvarpi}{\ensuremath{\ifthenelse{\boolean{uprightparticles}}{\upvarpi}{\varpi}}\xspace}
\DeclareRobustCommand{\Pvarrho}{\ensuremath{\ifthenelse{\boolean{uprightparticles}}{\upvarrho}{\varrho}}\xspace}
\DeclareRobustCommand{\Pvartheta}{\ensuremath{\ifthenelse{\boolean{uprightparticles}}{\upvartheta}{\vartheta}}\xspace}
\DeclareRobustCommand{\Pxi}{\ensuremath{\ifthenelse{\boolean{uprightparticles}}{\upxi}{\xi}}\xspace}
\DeclareRobustCommand{\Pzeta}{\ensuremath{\ifthenelse{\boolean{uprightparticles}}{\upzeta}{\zeta}}\xspace}

\DeclareRobustCommand{\antiP}[1]{{\ensuremath{\ifthenelse{\boolean{uprightparticles}}{\bar{#1}}{\overline{#1}}}}\xspace}
%%%%%%%%%%%%%%%%%%%%%%%%%%%%%%%%%%%%%%%%%%%%%%%
% Particles
\makeatletter
\ifcase \@ptsize \relax% 10pt
  \newcommand{\miniscule}{\@setfontsize\miniscule{4}{5}}% \tiny: 5/6
\or% 11pt
  \newcommand{\miniscule}{\@setfontsize\miniscule{5}{6}}% \tiny: 6/7
\or% 12pt
  \newcommand{\miniscule}{\@setfontsize\miniscule{5}{6}}% \tiny: 6/7
\fi
\makeatother


\DeclareRobustCommand{\optbar}[1]{\shortstack{{\miniscule (\rule[.5ex]{1.25em}{.18mm})}
  \\ [-.7ex] $#1$}}


%% Leptons

\let\emi\en
\def\electron   {{\ensuremath{\Pe}}\xspace}
\def\en         {{\ensuremath{\Pe^-}}\xspace}   % electron negative (\em is taken)
\def\ep         {{\ensuremath{\Pe^+}}\xspace}
\def\epm        {{\ensuremath{\Pe^\pm}}\xspace} 
\def\emp        {{\ensuremath{\Pe^\mp}}\xspace}
\def\epem       {{\ensuremath{\Pe^+\Pe^-}}\xspace}
%%%\def\ee         {\ensuremath{\Pe^-\Pe^-}\xspace}

\def\muon       {{\ensuremath{\Pmu}}\xspace}
\def\mup        {{\ensuremath{\Pmu^+}}\xspace}
\def\mun        {{\ensuremath{\Pmu^-}}\xspace} % muon negative (\mum is taken)
\def\mupm       {{\ensuremath{\Pmu^\pm}}\xspace}
\def\mumu       {{\ensuremath{\Pmu^+\Pmu^-}}\xspace}

\def\tauon      {{\ensuremath{\Ptau}}\xspace}
\def\taup       {{\ensuremath{\Ptau^+}}\xspace}
\def\taum       {{\ensuremath{\Ptau^-}}\xspace}
\def\taupm      {{\ensuremath{\Ptau^{\pm}}}\xspace}
\def\tautau     {{\ensuremath{\Ptau^+\Ptau^-}}\xspace}

\def\lepton     {{\ensuremath{\ell}}\xspace}
\def\ellm       {{\ensuremath{\ell^-}}\xspace}
\def\ellp       {{\ensuremath{\ell^+}}\xspace}
\def\ellpm      {{\ensuremath{\ell^{\pm}}}\xspace}
\def\ellell     {\ensuremath{\ell^+ \ell^-}\xspace}

\def\neu        {{\ensuremath{\Pnu}}\xspace}
\def\neub       {{\ensuremath{\overline{\Pnu}}}\xspace}
%%%\def\nuenueb    {\ensuremath{\neu\neub}\xspace}
\def\neue       {{\ensuremath{\neu_e}}\xspace}
\def\neueb      {{\ensuremath{\neub_e}}\xspace}
%%%\def\neueneueb  {\ensuremath{\neue\neueb}\xspace}
\def\neum       {{\ensuremath{\neu_\mu}}\xspace}
\def\neumb      {{\ensuremath{\neub_\mu}}\xspace}
%%%\def\neumneumb  {\ensuremath{\neum\neumb}\xspace}
\def\neut       {{\ensuremath{\neu_\tau}}\xspace}
\def\neutb      {{\ensuremath{\neub_\tau}}\xspace}
%%%\def\neutneutb  {\ensuremath{\neut\neutb}\xspace}
\def\neul       {{\ensuremath{\neu_\ell}}\xspace}
\def\neulb      {{\ensuremath{\neub_\ell}}\xspace}
%%%\def\neulneulb  {\ensuremath{\neul\neulb}\xspace}

%% Gauge bosons and scalars

\def\g      {{\ensuremath{\Pgamma}}\xspace}
\def\gs     {{\ensuremath{\Pgamma^{\ast}}}\xspace}
\def\H      {{\ensuremath{\PH^0}}\xspace}
\def\Hp     {{\ensuremath{\PH^+}}\xspace}
\def\Hm     {{\ensuremath{\PH^-}}\xspace}
\def\Hpm    {{\ensuremath{\PH^\pm}}\xspace}
\def\W      {{\ensuremath{\PW}}\xspace}
\def\Wp     {{\ensuremath{\PW^+}}\xspace}
\def\Wm     {{\ensuremath{\PW^-}}\xspace}
\def\Wpm    {{\ensuremath{\PW^\pm}}\xspace}
\def\Z      {{\ensuremath{\PZ}}\xspace}
\def\Zz     {{\ensuremath{\PZ^0}}\xspace}
\def\Zg     {{\ensuremath{\Zz/\Pgamma^\ast}}\xspace}

%% Quarks

\def\quark     {{\ensuremath{\Pq}}\xspace}
\def\quarkbar  {\antiP{\quark}}
\def\qqbar     {{\ensuremath{\quark\quarkbar}}\xspace}
\def\uquark    {{\ensuremath{\Pu}}\xspace}
\def\uquarkbar {\antiP{\uquark}}
\def\uubar     {{\ensuremath{\uquark\uquarkbar}}\xspace}
\def\dquark    {{\ensuremath{\Pd}}\xspace}
\def\dquarkbar {\antiP{\dquark}}
\def\ddbar     {{\ensuremath{\dquark\dquarkbar}}\xspace}
\def\squark    {{\ensuremath{\Ps}}\xspace}
\def\squarkbar {\antiP{\squark}}
\def\ssbar     {{\ensuremath{\squark\squarkbar}}\xspace}
\def\cquark    {{\ensuremath{\Pc}}\xspace}
\def\cquarkbar {\antiP{\cquark}}
\def\ccbar     {{\ensuremath{\cquark\cquarkbar}}\xspace}
\def\bquark    {{\ensuremath{\Pb}}\xspace}
\def\bquarkbar {\antiP{\bquark}}
\def\bbbar     {{\ensuremath{\bquark\bquarkbar}}\xspace}
\def\tquark    {{\ensuremath{\Pt}}\xspace}
\def\tquarkbar {\antiP{\tquark}}
\def\ttbar     {{\ensuremath{\tquark\tquarkbar}}\xspace}

%% Light mesons

\def\hadron {{\ensuremath{\Ph}}\xspace}
\def\pion   {{\ensuremath{\Ppi}}\xspace}
\def\piz    {{\ensuremath{\pion^0}}\xspace}
\def\pizs   {{\ensuremath{\pion^0\mbox\,\mathrm{s}}}\xspace}
\def\pip    {{\ensuremath{\pion^+}}\xspace}
\def\pim    {{\ensuremath{\pion^-}}\xspace}
\def\pipm   {{\ensuremath{\pion^\pm}}\xspace}
\def\pimp   {{\ensuremath{\pion^\mp}}\xspace}

\def\rhomeson {{\ensuremath{\Prho}}\xspace}
\def\rhoz     {{\ensuremath{\rhomeson^0}}\xspace}
\def\rhop     {{\ensuremath{\rhomeson^+}}\xspace}
\def\rhom     {{\ensuremath{\rhomeson^-}}\xspace}
\def\rhopm    {{\ensuremath{\rhomeson^\pm}}\xspace}
\def\rhomp    {{\ensuremath{\rhomeson^\mp}}\xspace}

\def\kaon    {{\ensuremath{\PK}}\xspace}
%%% do NOT use ensuremath here
\def\Kbar    {{\kern 0.2em\overline{\kern -0.2em \PK}{}}\xspace} % do_not_list
\def\Kb      {{\ensuremath{\Kbar}}\xspace}
\def\KorKbar {\kern 0.18em\optbar{\kern -0.18em K}{}\xspace} % do_not_list
\def\Kz      {{\ensuremath{\kaon^0}}\xspace}
\def\Kzb     {{\ensuremath{\Kbar{}^0}}\xspace}
\def\Kp      {{\ensuremath{\kaon^+}}\xspace}
\def\Km      {{\ensuremath{\kaon^-}}\xspace}
\def\Kpm     {{\ensuremath{\kaon^\pm}}\xspace}
\def\Kmp     {{\ensuremath{\kaon^\mp}}\xspace}
\def\KS      {{\ensuremath{\kaon^0_{\mathrm{ \scriptscriptstyle S}}}}\xspace}
\def\KL      {{\ensuremath{\kaon^0_{\mathrm{ \scriptscriptstyle L}}}}\xspace}
\def\Kstarz  {{\ensuremath{\kaon^{*0}}}\xspace}
\def\Kstarzb {{\ensuremath{\Kbar{}^{*0}}}\xspace}
\def\Kstar   {{\ensuremath{\kaon^*}}\xspace}
\def\Kstarb  {{\ensuremath{\Kbar{}^*}}\xspace}
\def\Kstarp  {{\ensuremath{\kaon^{*+}}}\xspace}
\def\Kstarm  {{\ensuremath{\kaon^{*-}}}\xspace}
\def\Kstarpm {{\ensuremath{\kaon^{*\pm}}}\xspace}
\def\Kstarmp {{\ensuremath{\kaon^{*\mp}}}\xspace}

\newcommand{\etaz}{\ensuremath{\Peta}\xspace}
\newcommand{\etapr}{\ensuremath{\Peta^{\prime}}\xspace}
\newcommand{\phiz}{\ensuremath{\Pphi}\xspace}
\newcommand{\omegaz}{\ensuremath{\Pomega}\xspace}

%% Heavy mesons

%%% do NOT use ensuremath here
\def\Dbar    {{\kern 0.2em\overline{\kern -0.2em \PD}{}}\xspace} % do_not_list
\def\D       {{\ensuremath{\PD}}\xspace}
\def\Db      {{\ensuremath{\Dbar}}\xspace}
\def\DorDbar {\kern 0.18em\optbar{\kern -0.18em D}{}\xspace}
\def\Dz      {{\ensuremath{\D^0}}\xspace}
\def\Dzb     {{\ensuremath{\Dbar{}^0}}\xspace}
\def\Dp      {{\ensuremath{\D^+}}\xspace}
\def\Dm      {{\ensuremath{\D^-}}\xspace}
\def\Dpm     {{\ensuremath{\D^\pm}}\xspace}
\def\Dmp     {{\ensuremath{\D^\mp}}\xspace}
\def\Dstar   {{\ensuremath{\D^*}}\xspace}
\def\Dstarb  {{\ensuremath{\Dbar{}^*}}\xspace}
\def\Dstarz  {{\ensuremath{\D^{*0}}}\xspace}
\def\Dstarzb {{\ensuremath{\Dbar{}^{*0}}}\xspace}
\def\Dstarp  {{\ensuremath{\D^{*+}}}\xspace}
\def\Dstarm  {{\ensuremath{\D^{*-}}}\xspace}
\def\Dstarpm {{\ensuremath{\D^{*\pm}}}\xspace}
\def\Dstarmp {{\ensuremath{\D^{*\mp}}}\xspace}
\def\Ds      {{\ensuremath{\D^+_\squark}}\xspace}
\def\Dsp     {{\ensuremath{\D^+_\squark}}\xspace}
\def\Dsm     {{\ensuremath{\D^-_\squark}}\xspace}
\def\Dspm    {{\ensuremath{\D^{\pm}_\squark}}\xspace}
\def\Dsmp    {{\ensuremath{\D^{\mp}_\squark}}\xspace}
\def\Dss     {{\ensuremath{\D^{*+}_\squark}}\xspace}
\def\Dssp    {{\ensuremath{\D^{*+}_\squark}}\xspace}
\def\Dssm    {{\ensuremath{\D^{*-}_\squark}}\xspace}
\def\Dsspm   {{\ensuremath{\D^{*\pm}_\squark}}\xspace}
\def\Dssmp   {{\ensuremath{\D^{*\mp}_\squark}}\xspace}

%%% do NOT use ensuremath here
\def\Bbar    {{\ensuremath{\kern 0.18em\overline{\kern -0.18em \PB}{}}}\xspace}
\def\Bb      {{\ensuremath{\Bbar}}\xspace}
\def\BorBbar {\kern 0.18em\optbar{\kern -0.18em B}{}\xspace}
\def\Bz      {{\ensuremath{\PB^0}}\xspace}
\def\Bzb     {{\ensuremath{\Bbar{}^0}}\xspace}
\def\Bu      {{\ensuremath{\PB^+}}\xspace}
\def\Bub     {{\ensuremath{\PB^-}}\xspace}
\def\Bp      {{\ensuremath{\Bu}}\xspace}
\def\Bm      {{\ensuremath{\Bub}}\xspace}
\def\Bpm     {{\ensuremath{\PB^\pm}}\xspace}
\def\Bmp     {{\ensuremath{\PB^\mp}}\xspace}
\def\Bd      {{\ensuremath{\PB^0}}\xspace}
\def\Bs      {{\ensuremath{\PB^0_\squark}}\xspace}
\def\Bds     {{\ensuremath{\PB^0_{(\squark)}}}\xspace}
\def\Bsb     {{\ensuremath{\Bbar{}^0_\squark}}\xspace}
\def\Bdb     {{\ensuremath{\Bbar{}^0}}\xspace}
\def\Bc      {{\ensuremath{\PB_\cquark^+}}\xspace}
\def\Bcp     {{\ensuremath{\PB_\cquark^+}}\xspace}
\def\Bcm     {{\ensuremath{\PB_\cquark^-}}\xspace}
\def\Bcpm    {{\ensuremath{\PB_\cquark^\pm}}\xspace}

%% Onia

\def\jpsi     {{\ensuremath{{\PJ\mskip -3mu/\mskip -2mu\Ppsi\mskip 2mu}}}\xspace}
\def\psitwos  {{\ensuremath{\Ppsi{(2S)}}}\xspace}
\def\psiprpr  {{\ensuremath{\Ppsi(3770)}}\xspace}
\def\etac     {{\ensuremath{\Peta_\cquark}}\xspace}
\def\chiczero {{\ensuremath{\Pchi_{\cquark 0}}}\xspace}
\def\chicone  {{\ensuremath{\Pchi_{\cquark 1}}}\xspace}
\def\chictwo  {{\ensuremath{\Pchi_{\cquark 2}}}\xspace}
  %\mathchardef\Upsilon="7107
  \def\Y#1S{\ensuremath{\PUpsilon{(#1S)}}\xspace}% no space before {...}!
\def\OneS  {{\Y1S}}
\def\TwoS  {{\Y2S}}
\def\ThreeS{{\Y3S}}
\def\FourS {{\Y4S}}
\def\FiveS {{\Y5S}}
\def\YnS   {{\ensuremath{\PUpsilon(n\mathrm{S})}}\xspace}

\def\chic  {{\ensuremath{\Pchi_{c}}}\xspace}

%% Baryons

\def\proton      {{\ensuremath{\Pp}}\xspace}
\def\antiproton  {\antiP{\proton}}
\def\neutron     {{\ensuremath{\Pn}}\xspace}
\def\antineutron {\antiP{\neutron}}
\def\Deltares    {{\ensuremath{\PDelta}}\xspace}
\def\Deltaresbar {{\ensuremath{\overline \Deltares}}\xspace}
\def\Xires       {{\ensuremath{\PXi}}\xspace}
\def\Xiresbar    {{\ensuremath{\overline \Xires}}\xspace}
\def\Lz          {{\ensuremath{\PLambda}}\xspace}
\def\Lzz         {{\ensuremath{\PLambda^0}}\xspace}
\def\Lbar        {{\ensuremath{\kern 0.1em\overline{\kern -0.1em\PLambda}}}\xspace}
\def\LorLbar    {\kern 0.18em\optbar{\kern -0.18em \PLambda}{}\xspace}
\def\Lambdares   {{\ensuremath{\PLambda}}\xspace}
\def\Lambdaresbar{{\ensuremath{\Lbar}}\xspace}
\def\Sigmares    {{\ensuremath{\PSigma}}\xspace}
\def\Sigmaresbar {{\ensuremath{\overline \Sigmares}}\xspace}
\def\Sigmaresbarz {{\ensuremath{\Sigmaresbar{}^0}}\xspace}
\def\Omegares    {{\ensuremath{\POmega}}\xspace}
\def\Omegaresbar {{\ensuremath{\overline \POmega}}\xspace}

\def\ppcol       {{\ensuremath{\proton\proton}}\xspace}
\def\ppbar       {{\ensuremath{\proton\antiproton}}\xspace}

%%% do NOT use ensuremath here
 % \def\Deltabar{\kern 0.25em\overline{\kern -0.25em \Deltares}{}\xspace}
 % \def\Sigbar{\kern 0.2em\overline{\kern -0.2em \Sigma}{}\xspace}
 % \def\Xibar{\kern 0.2em\overline{\kern -0.2em \Xi}{}\xspace}
 % \def\Obar{\kern 0.2em\overline{\kern -0.2em \Omega}{}\xspace}
 % \def\Nbar{\kern 0.2em\overline{\kern -0.2em N}{}\xspace}
 % \def\Xb{\kern 0.2em\overline{\kern -0.2em X}{}\xspace}

\def\Lb      {{\ensuremath{\Lz^0_\bquark}}\xspace}
\def\Lbbar   {{\ensuremath{\Lbar{}^0_\bquark}}\xspace}
\def\Lc      {{\ensuremath{\Lz^+_\cquark}}\xspace}
\def\Lcbar   {{\ensuremath{\Lbar{}^-_\cquark}}\xspace}
\def\Xib     {{\ensuremath{\Xires_\bquark}}\xspace}
\def\Xibz    {{\ensuremath{\Xires^0_\bquark}}\xspace}
\def\Xibm    {{\ensuremath{\Xires^-_\bquark}}\xspace}
\def\Xibbar  {{\ensuremath{\Xiresbar{}_\bquark}}\xspace}
\def\Xibbarz {{\ensuremath{\Xiresbar{}_\bquark^0}}\xspace}
\def\Xibbarp {{\ensuremath{\Xiresbar{}_\bquark^+}}\xspace}
\def\Xic     {{\ensuremath{\Xires_\cquark}}\xspace}
\def\Xicz    {{\ensuremath{\Xires^0_\cquark}}\xspace}
\def\Xicp    {{\ensuremath{\Xires^+_\cquark}}\xspace}
\def\Xicbar  {{\ensuremath{\Xiresbar{}_\cquark}}\xspace}
\def\Xicbarz {{\ensuremath{\Xiresbar{}_\cquark^0}}\xspace}
\def\Xicbarm {{\ensuremath{\Xiresbar{}_\cquark^-}}\xspace}
\def\Omegac    {{\ensuremath{\Omegares^0_\cquark}}\xspace}
\def\Omegacbar {{\ensuremath{\Omegaresbar{}_\cquark^0}}\xspace}
\def\Omegab    {{\ensuremath{\Omegares^-_\bquark}}\xspace}
\def\Omegabbar {{\ensuremath{\Omegaresbar{}_\bquark^+}}\xspace}

%%%%%%%%%%%%%%%%%%
% Physics symbols
%%%%%%%%%%%%%%%%%

%% Decays
\def\BF         {{\ensuremath{\mathcal{B}}}\xspace}
\def\BRvis      {{\ensuremath{\BR_{\mathrm{{vis}}}}}}
\def\BR         {\BF}
\newcommand{\decay}[2]{\ensuremath{#1\!\to #2}\xspace}         % {\Pa}{\Pb \Pc}
\def\ra                 {\ensuremath{\rightarrow}\xspace}
\def\to                 {\ensuremath{\rightarrow}\xspace}

%% Lifetimes
\newcommand{\tauBs}{{\ensuremath{\tau_{\Bs}}}\xspace}
\newcommand{\tauBd}{{\ensuremath{\tau_{\Bd}}}\xspace}
\newcommand{\tauBz}{{\ensuremath{\tau_{\Bz}}}\xspace}
\newcommand{\tauBu}{{\ensuremath{\tau_{\Bp}}}\xspace}
\newcommand{\tauDp}{{\ensuremath{\tau_{\Dp}}}\xspace}
\newcommand{\tauDz}{{\ensuremath{\tau_{\Dz}}}\xspace}
\newcommand{\tauL}{{\ensuremath{\tau_{\mathrm{ L}}}}\xspace}
\newcommand{\tauH}{{\ensuremath{\tau_{\mathrm{ H}}}}\xspace}

%% Masses
\newcommand{\mBd}{{\ensuremath{m_{\Bd}}}\xspace}
\newcommand{\mBp}{{\ensuremath{m_{\Bp}}}\xspace}
\newcommand{\mBs}{{\ensuremath{m_{\Bs}}}\xspace}
\newcommand{\mBc}{{\ensuremath{m_{\Bc}}}\xspace}
\newcommand{\mH}{{\ensuremath{m_{\PH}}}\xspace}
\newcommand{\mLb}{{\ensuremath{m_{\Lb}}}\xspace}
\newcommand{\mtquark}{{\ensuremath{m_{\tquark}}}\xspace}
\newcommand{\mW}{{\ensuremath{m_{\PW}}}\xspace}
\newcommand{\mZ}{{\ensuremath{m_{\PZ}}}\xspace}

%% EW theory, groups
\def\grpsuthree {{\ensuremath{\mathrm{SU}(3)}}\xspace}
\def\grpsutw    {{\ensuremath{\mathrm{SU}(2)}}\xspace}
\def\grpuone    {{\ensuremath{\mathrm{U}(1)}}\xspace}

\def\ssqtw   {{\ensuremath{\sin^{2}\!\theta_{\mathrm{W}}}}\xspace}
\def\ssqtlef {{\ensuremath{{\sin}^{2}\theta_{\mathrm{eff}}^{\mathrm{lept}}}}\xspace}
\def\csqtw   {{\ensuremath{\cos^{2}\!\theta_{\mathrm{W}}}}\xspace}
\def\stw     {{\ensuremath{\sin\theta_{\mathrm{W}}}}\xspace}
\def\ctw     {{\ensuremath{\cos\theta_{\mathrm{W}}}}\xspace}
\def\ssqtwef {{\ensuremath{{\sin}^{2}\theta_{\mathrm{W}}^{\mathrm{eff}}}}\xspace}
\def\csqtwef {{\ensuremath{{\cos}^{2}\theta_{\mathrm{W}}^{\mathrm{eff}}}}\xspace}
\def\stwef   {{\ensuremath{\sin\theta_{\mathrm{W}}^{\mathrm{eff}}}}\xspace}
\def\ctwef   {{\ensuremath{\cos\theta_{\mathrm{W}}^{\mathrm{eff}}}}\xspace}
\def\gv      {{\ensuremath{g_{\mbox{\tiny V}}}}\xspace}
\def\ga      {{\ensuremath{g_{\mbox{\tiny A}}}}\xspace}

\def\order   {{\ensuremath{\mathcal{O}}}\xspace}
\def\ordalph {{\ensuremath{\mathcal{O}(\alpha)}}\xspace}
\def\ordalsq {{\ensuremath{\mathcal{O}(\alpha^{2})}}\xspace}
\def\ordalcb {{\ensuremath{\mathcal{O}(\alpha^{3})}}\xspace}

%% QCD parameters
\newcommand{\as}{{\ensuremath{\alpha_s}}\xspace}
\newcommand{\MSb}{{\ensuremath{\overline{\mathrm{MS}}}}\xspace}
\newcommand{\lqcd}{{\ensuremath{\Lambda_{\mathrm{QCD}}}}\xspace}
\def\qsq       {{\ensuremath{q^2}}\xspace}

%% CKM, CP violation

\def\eps   {{\ensuremath{\varepsilon}}\xspace}
\def\epsK  {{\ensuremath{\varepsilon_K}}\xspace}
\def\epsB  {{\ensuremath{\varepsilon_B}}\xspace}
\def\epsp  {{\ensuremath{\varepsilon^\prime_K}}\xspace}

\def\CP                {{\ensuremath{C\!P}}\xspace}
\def\CPT               {{\ensuremath{C\!PT}}\xspace}

\def\rhobar {{\ensuremath{\overline \rho}}\xspace}
\def\etabar {{\ensuremath{\overline \eta}}\xspace}

\def\Vud  {{\ensuremath{V_{\uquark\dquark}}}\xspace}
\def\Vcd  {{\ensuremath{V_{\cquark\dquark}}}\xspace}
\def\Vtd  {{\ensuremath{V_{\tquark\dquark}}}\xspace}
\def\Vus  {{\ensuremath{V_{\uquark\squark}}}\xspace}
\def\Vcs  {{\ensuremath{V_{\cquark\squark}}}\xspace}
\def\Vts  {{\ensuremath{V_{\tquark\squark}}}\xspace}
\def\Vub  {{\ensuremath{V_{\uquark\bquark}}}\xspace}
\def\Vcb  {{\ensuremath{V_{\cquark\bquark}}}\xspace}
\def\Vtb  {{\ensuremath{V_{\tquark\bquark}}}\xspace}
\def\Vuds  {{\ensuremath{V_{\uquark\dquark}^\ast}}\xspace}
\def\Vcds  {{\ensuremath{V_{\cquark\dquark}^\ast}}\xspace}
\def\Vtds  {{\ensuremath{V_{\tquark\dquark}^\ast}}\xspace}
\def\Vuss  {{\ensuremath{V_{\uquark\squark}^\ast}}\xspace}
\def\Vcss  {{\ensuremath{V_{\cquark\squark}^\ast}}\xspace}
\def\Vtss  {{\ensuremath{V_{\tquark\squark}^\ast}}\xspace}
\def\Vubs  {{\ensuremath{V_{\uquark\bquark}^\ast}}\xspace}
\def\Vcbs  {{\ensuremath{V_{\cquark\bquark}^\ast}}\xspace}
\def\Vtbs  {{\ensuremath{V_{\tquark\bquark}^\ast}}\xspace}

%% Oscillations

\newcommand{\dm}{{\ensuremath{\Delta m}}\xspace}
\newcommand{\dms}{{\ensuremath{\Delta m_{\squark}}}\xspace}
\newcommand{\dmd}{{\ensuremath{\Delta m_{\dquark}}}\xspace}
\newcommand{\DG}{{\ensuremath{\Delta\Gamma}}\xspace}
\newcommand{\DGs}{{\ensuremath{\Delta\Gamma_{\squark}}}\xspace}
\newcommand{\DGd}{{\ensuremath{\Delta\Gamma_{\dquark}}}\xspace}
\newcommand{\Gs}{{\ensuremath{\Gamma_{\squark}}}\xspace}
\newcommand{\Gd}{{\ensuremath{\Gamma_{\dquark}}}\xspace}
\newcommand{\MBq}{{\ensuremath{M_{\PB_\quark}}}\xspace}
\newcommand{\DGq}{{\ensuremath{\Delta\Gamma_{\quark}}}\xspace}
\newcommand{\Gq}{{\ensuremath{\Gamma_{\quark}}}\xspace}
\newcommand{\dmq}{{\ensuremath{\Delta m_{\quark}}}\xspace}
\newcommand{\GL}{{\ensuremath{\Gamma_{\mathrm{ L}}}}\xspace}
\newcommand{\GH}{{\ensuremath{\Gamma_{\mathrm{ H}}}}\xspace}
\newcommand{\DGsGs}{{\ensuremath{\Delta\Gamma_{\squark}/\Gamma_{\squark}}}\xspace}
\newcommand{\Delm}{{\mbox{$\Delta m $}}\xspace}
\newcommand{\ACP}{{\ensuremath{{\mathcal{A}}^{\CP}}}\xspace}
\newcommand{\Adir}{{\ensuremath{{\mathcal{A}}^{\mathrm{ dir}}}}\xspace}
\newcommand{\Amix}{{\ensuremath{{\mathcal{A}}^{\mathrm{ mix}}}}\xspace}
\newcommand{\ADelta}{{\ensuremath{{\mathcal{A}}^\Delta}}\xspace}
\newcommand{\phid}{{\ensuremath{\phi_{\dquark}}}\xspace}
\newcommand{\sinphid}{{\ensuremath{\sin\!\phid}}\xspace}
\newcommand{\phis}{{\ensuremath{\phi_{\squark}}}\xspace}
\newcommand{\betas}{{\ensuremath{\beta_{\squark}}}\xspace}
\newcommand{\sbetas}{{\ensuremath{\sigma(\beta_{\squark})}}\xspace}
\newcommand{\stbetas}{{\ensuremath{\sigma(2\beta_{\squark})}}\xspace}
\newcommand{\stphis}{{\ensuremath{\sigma(\phi_{\squark})}}\xspace}
\newcommand{\sinphis}{{\ensuremath{\sin\!\phis}}\xspace}

%% Tagging
\newcommand{\edet}{{\ensuremath{\varepsilon_{\mathrm{ det}}}}\xspace}
\newcommand{\erec}{{\ensuremath{\varepsilon_{\mathrm{ rec/det}}}}\xspace}
\newcommand{\esel}{{\ensuremath{\varepsilon_{\mathrm{ sel/rec}}}}\xspace}
\newcommand{\etrg}{{\ensuremath{\varepsilon_{\mathrm{ trg/sel}}}}\xspace}
\newcommand{\etot}{{\ensuremath{\varepsilon_{\mathrm{ tot}}}}\xspace}

\newcommand{\mistag}{\ensuremath{\omega}\xspace}
\newcommand{\wcomb}{\ensuremath{\omega^{\mathrm{comb}}}\xspace}
\newcommand{\etag}{{\ensuremath{\varepsilon_{\mathrm{tag}}}}\xspace}
\newcommand{\etagcomb}{{\ensuremath{\varepsilon_{\mathrm{tag}}^{\mathrm{comb}}}}\xspace}
\newcommand{\effeff}{\ensuremath{\varepsilon_{\mathrm{eff}}}\xspace}
\newcommand{\effeffcomb}{\ensuremath{\varepsilon_{\mathrm{eff}}^{\mathrm{comb}}}\xspace}
\newcommand{\efftag}{{\ensuremath{\etag(1-2\omega)^2}}\xspace}
\newcommand{\effD}{{\ensuremath{\etag D^2}}\xspace}

\newcommand{\etagprompt}{{\ensuremath{\varepsilon_{\mathrm{ tag}}^{\mathrm{Pr}}}}\xspace}
\newcommand{\etagLL}{{\ensuremath{\varepsilon_{\mathrm{ tag}}^{\mathrm{LL}}}}\xspace}

%% Key decay channels

\def\BdToKstmm    {\decay{\Bd}{\Kstarz\mup\mun}}
\def\BdbToKstmm   {\decay{\Bdb}{\Kstarzb\mup\mun}}

\def\BsToJPsiPhi  {\decay{\Bs}{\jpsi\phi}}
\def\BdToJPsiKst  {\decay{\Bd}{\jpsi\Kstarz}}
\def\BdbToJPsiKst {\decay{\Bdb}{\jpsi\Kstarzb}}

\def\BsPhiGam     {\decay{\Bs}{\phi \g}}
\def\BdKstGam     {\decay{\Bd}{\Kstarz \g}}

\def\BTohh        {\decay{\PB}{\Ph^+ \Ph'^-}}
\def\BdTopipi     {\decay{\Bd}{\pip\pim}}
\def\BdToKpi      {\decay{\Bd}{\Kp\pim}}
\def\BsToKK       {\decay{\Bs}{\Kp\Km}}
\def\BsTopiK      {\decay{\Bs}{\pip\Km}}

%% Rare decays
\def\BdKstee  {\decay{\Bd}{\Kstarz\epem}}
\def\BdbKstee {\decay{\Bdb}{\Kstarzb\epem}}
\def\bsll     {\decay{\bquark}{\squark \ell^+ \ell^-}}
\def\AFB      {\ensuremath{A_{\mathrm{FB}}}\xspace}
\def\FL       {\ensuremath{F_{\mathrm{L}}}\xspace}
\def\AT#1     {\ensuremath{A_{\mathrm{T}}^{#1}}\xspace}           % 2
\def\btosgam  {\decay{\bquark}{\squark \g}}
\def\btodgam  {\decay{\bquark}{\dquark \g}}
\def\Bsmm     {\decay{\Bs}{\mup\mun}}
\def\Bdmm     {\decay{\Bd}{\mup\mun}}
\def\ctl       {\ensuremath{\cos{\theta_\ell}}\xspace}
\def\ctk       {\ensuremath{\cos{\theta_K}}\xspace}

%% Wilson coefficients and operators
\def\C#1      {\ensuremath{\mathcal{C}_{#1}}\xspace}                       % 9
\def\Cp#1     {\ensuremath{\mathcal{C}_{#1}^{'}}\xspace}                    % 7
\def\Ceff#1   {\ensuremath{\mathcal{C}_{#1}^{\mathrm{(eff)}}}\xspace}        % 9  
\def\Cpeff#1  {\ensuremath{\mathcal{C}_{#1}^{'\mathrm{(eff)}}}\xspace}       % 7
\def\Ope#1    {\ensuremath{\mathcal{O}_{#1}}\xspace}                       % 2
\def\Opep#1   {\ensuremath{\mathcal{O}_{#1}^{'}}\xspace}                    % 7

%% Charm

\def\xprime     {\ensuremath{x^{\prime}}\xspace}
\def\yprime     {\ensuremath{y^{\prime}}\xspace}
\def\ycp        {\ensuremath{y_{\CP}}\xspace}
\def\agamma     {\ensuremath{A_{\Gamma}}\xspace}
%%%\def\kpi        {\ensuremath{\PK\Ppi}\xspace}
%%%\def\kk         {\ensuremath{\PK\PK}\xspace}
%%%\def\dkpi       {\decay{\PD}{\PK\Ppi}}
%%%\def\dkk        {\decay{\PD}{\PK\PK}}
\def\dkpicf     {\decay{\Dz}{\Km\pip}}

%% QM
\newcommand{\bra}[1]{\ensuremath{\langle #1|}}             % {a}
\newcommand{\ket}[1]{\ensuremath{|#1\rangle}}              % {b}
\newcommand{\braket}[2]{\ensuremath{\langle #1|#2\rangle}} % {a}{b}

%%%%%%%%%%%%%%%%%%%%%%%%%%%%%%%%%%%%%%%%%%%%%%%%%%
% Units - use the siunitx commands for typesetting
%%%%%%%%%%%%%%%%%%%%%%%%%%%%%%%%%%%%%%%%%%%%%%%%%%

\DeclareSIUnit\clight{\text{\ensuremath{c}}}
\DeclareSIUnit\micron{\micro\metre}
\DeclareSIUnit\mrad{\milli\radian}
\DeclareSIUnit\gauss{G}

%% Energy, momentum, mass
\DeclareSIUnit\meV{\milli\eV}
\DeclareSIUnit\keV{\kilo\eV}
\DeclareSIUnit\MeV{\mega\eV}
\DeclareSIUnit\GeV{\giga\eV}
\DeclareSIUnit\TeV{\tera\eV}

\DeclareSIUnit[per-mode=symbol]\MeVc{\MeV\!\per\clight}
\DeclareSIUnit[per-mode=symbol]\GeVc{\GeV\!\per\clight}

\DeclareSIUnit[per-mode=symbol]\MeVcc{\MeV\!\per\clight\squared}
\DeclareSIUnit[per-mode=symbol]\GeVcc{\GeV\!\per\clight\squared}
\DeclareSIUnit[per-mode=symbol]\GeVGeVcccc{\GeV\squared\!\per\clight^{4}}

%% Cross-section and integrated luminosity
\DeclareSIUnit\mb{\micro\barn}
\DeclareSIUnit\nb{\nano\barn}
\DeclareSIUnit\pb{\pico\barn}
\DeclareSIUnit\fb{\femto\barn}
\DeclareSIUnit\ab{\atto\barn}
\DeclareSIUnit\zb{\zepto\barn}
\DeclareSIUnit\yb{\yocto\barn}

\DeclareSIUnit\invnb{\per\nano\barn}
\DeclareSIUnit\invpb{\per\pico\barn}
\DeclareSIUnit\invfb{\per\femto\barn}
\DeclareSIUnit\invab{\per\atto\barn}

\DeclareSIUnit\Hzpercmsq{\per\centi\metre\squared\per\second}

%% Material lengths, radiation
\DeclareSIUnit\Xrad{\text{\ensuremath{X_{0}}}}
\DeclareSIUnit\NIL{\text{\ensuremath{\lambda_{\text{int}}}}}
\DeclareSIUnit\mip{MIP}
  % TODO convert the rest of these
\def\neutroneq {\ensuremath{\mathrm{ \,n_{eq}}}\xspace}
\def\neqcmcm {\ensuremath{\mathrm{ \,n_{eq} / cm^2}}\xspace}
\def\kRad {\ensuremath{\mathrm{ \,kRad}}\xspace}
\def\MRad {\ensuremath{\mathrm{ \,MRad}}\xspace}
\def\ci {\ensuremath{\mathrm{ \,Ci}}\xspace}
\def\mci {\ensuremath{\mathrm{ \,mCi}}\xspace}

%% Uncertainties
\def\sx    {\ensuremath{\sigma_x}\xspace}    
\def\sy    {\ensuremath{\sigma_y}\xspace}   
\def\sz    {\ensuremath{\sigma_z}\xspace}    

\newcommand{\stat}{\ensuremath{\mathrm{\,(stat)}}\xspace}
\newcommand{\syst}{\ensuremath{\mathrm{\,(syst)}}\xspace}
\newcommand{\lumi}{\ensuremath{\mathrm{\,(lumi)}}\xspace}

%% Maths

\def\order{{\ensuremath{\mathcal{O}}}\xspace}
\newcommand{\chisq}{\ensuremath{\chi^2}\xspace}
\newcommand{\chisqndf}{\ensuremath{\chi^2/\mathrm{ndf}}\xspace}
\newcommand{\chisqip}{\ensuremath{\chi^2_{\text{IP}}}\xspace}
\newcommand{\chisqvs}{\ensuremath{\chi^2_{\text{VS}}}\xspace}
\newcommand{\chisqvtx}{\ensuremath{\chi^2_{\text{vtx}}}\xspace}
\newcommand{\chisqvtxndf}{\ensuremath{\chi^2_{\text{vtx}}/\mathrm{ndf}}\xspace}

\def\deriv {\ensuremath{\mathrm{d}}}

\def\gsim{{~\raise.15em\hbox{$>$}\kern-.85em
          \lower.35em\hbox{$\sim$}~}\xspace}
\def\lsim{{~\raise.15em\hbox{$<$}\kern-.85em
          \lower.35em\hbox{$\sim$}~}\xspace}

\newcommand{\mean}[1]{\ensuremath{\left\langle #1 \right\rangle}} % {x}
\newcommand{\abs}[1]{\ensuremath{\left\|#1\right\|}} % {x}
\newcommand{\Real}{\ensuremath{\mathcal{R}e}\xspace}
\newcommand{\Imag}{\ensuremath{\mathcal{I}m}\xspace}

\def\PDF {PDF\xspace}
\def\nnpdf  {\mbox{NNPDF}\xspace}

\def\sPlot{\mbox{\em sPlot}\xspace}
%%%\def\sWeight{\mbox{\em sWeight}\xspace}

%%%%%%%%%%%%%%%%%%%%%%%%%%%%%%%%%%%%%%%%%%%%%%%%%%
% Kinematics
%%%%%%%%%%%%%%%%%%%%%%%%%%%%%%%%%%%%%%%%%%%%%%%%%%

%% Energy, Momenta
\def\Ebeam {\ensuremath{E_{\mbox{\tiny BEAM}}}\xspace}
\def\sqs   {\ensuremath{\protect\sqrt{s}}\xspace}

\def\ptot       {\mbox{$p$}\xspace}
\def\pt         {\mbox{$p_{\mathrm{ T}}$}\xspace}
\def\et         {\mbox{$E_{\mathrm{ T}}$}\xspace}
\def\etmiss     {\mbox{$E_{\mathrm{ T}}^{\mathrm{miss}}$}\xspace}
\def\mt         {\mbox{$M_{\mathrm{ T}}$}\xspace}
\def\dpp        {\ensuremath{\Delta p/p}\xspace}
\def\msq        {\ensuremath{m^2}\xspace}
\newcommand{\dedx}{\ensuremath{\mathrm{d}\hspace{-0.1em}E/\mathrm{d}x}\xspace}

%% PID

\def\dllkpi     {\ensuremath{\mathrm{DLL}_{\kaon\pion}}\xspace}
\def\dllppi     {\ensuremath{\mathrm{DLL}_{\proton\pion}}\xspace}
\def\dllepi     {\ensuremath{\mathrm{DLL}_{\electron\pion}}\xspace}
\def\dllmupi    {\ensuremath{\mathrm{DLL}_{\muon\pi}}\xspace}

%% Geometry
%%%\def\mphi       {\mbox{$\phi$}\xspace}
%%%\def\mtheta     {\mbox{$\theta$}\xspace}
%%%\def\ctheta     {\mbox{$\cos\theta$}\xspace}
%%%\def\stheta     {\mbox{$\sin\theta$}\xspace}
%%%\def\ttheta     {\mbox{$\tan\theta$}\xspace}

%% Accelerator
\def\betastar {\ensuremath{\beta^*}}
\newcommand{\lum} {\ensuremath{\mathcal{L}}\xspace}
\newcommand{\intlum}[1]{\ensuremath{\int\lum=#1}\xspace}  % { \SI{2}{\invfb} }

%%%%%%%%%%%%%%%%%%%%%%%%%%%%%%%%%%%%%%%%%%%%%%%%%%%%%%%%%%%%%%%%%%%%
% Software
%%%%%%%%%%%%%%%%%%%%%%%%%%%%%%%%%%%%%%%%%%%%%%%%%%%%%%%%%%%%%%%%%%%%

%% Programs
%%%\def\ansys      {\mbox{\textsc{Ansys}}\xspace}
\def\bcvegpy    {\mbox{\textsc{Bcvegpy}}\xspace}
\def\boole      {\mbox{\textsc{Boole}}\xspace}
\def\brunel     {\mbox{\textsc{Brunel}}\xspace}
\def\davinci    {\mbox{\textsc{DaVinci}}\xspace}
\def\dirac      {\mbox{\textsc{Dirac}}\xspace}
%%%\def\erasmus    {\mbox{\textsc{Erasmus}}\xspace}
\def\evtgen     {\mbox{\textsc{EvtGen}}\xspace}
\def\fewz       {\mbox{\textsc{Fewz}}\xspace}
\def\fluka      {\mbox{\textsc{Fluka}}\xspace}
\def\ganga      {\mbox{\textsc{Ganga}}\xspace}
%%%\def\garfield   {\mbox{\textsc{Garfield}}\xspace}
\def\gaudi      {\mbox{\textsc{Gaudi}}\xspace}
\def\gauss      {\mbox{\textsc{Gauss}}\xspace}
\def\geant      {\mbox{\textsc{Geant4}}\xspace}
\def\hepmc      {\mbox{\textsc{HepMC}}\xspace}
\def\herwig     {\mbox{\textsc{Herwig}}\xspace}
\def\moore      {\mbox{\textsc{Moore}}\xspace}
\def\neurobayes {\mbox{\textsc{NeuroBayes}}\xspace}
\def\photos     {\mbox{\textsc{Photos}}\xspace}
\def\powheg     {\mbox{\textsc{Powheg}}\xspace}
%%%\def\pyroot     {\mbox{\textsc{PyRoot}}\xspace}
\def\pythia     {\mbox{\textsc{Pythia}}\xspace}
\def\resbos     {\mbox{\textsc{ResBos}}\xspace}
\def\roofit     {\mbox{\textsc{RooFit}}\xspace}
\def\root       {\mbox{\textsc{Root}}\xspace}
\def\spice      {\mbox{\textsc{Spice}}\xspace}
%%%\def\tosca      {\mbox{\textsc{Tosca}}\xspace}
\def\urania     {\mbox{\textsc{Urania}}\xspace}

%% Languages
\def\cpp        {\mbox{\textsc{C\raisebox{0.1em}{{\footnotesize{++}}}}}\xspace}
%%%\def\python     {\mbox{\textsc{Python}}\xspace}
\def\ruby       {\mbox{\textsc{Ruby}}\xspace}
\def\fortran    {\mbox{\textsc{Fortran}}\xspace}
\def\svn        {\mbox{\textsc{SVN}}\xspace}

%% Data processing
\def\kbytes     {\ensuremath{{\mathrm{ \,kbytes}}}\xspace}
\def\kbsps      {\ensuremath{{\mathrm{ \,kbytes/s}}}\xspace}
\def\kbits      {\ensuremath{{\mathrm{ \,kbits}}}\xspace}
\def\kbsps      {\ensuremath{{\mathrm{ \,kbits/s}}}\xspace}
\def\mbsps      {\ensuremath{{\mathrm{ \,Mbits/s}}}\xspace}
\def\mbytes     {\ensuremath{{\mathrm{ \,Mbytes}}}\xspace}
\def\mbps       {\ensuremath{{\mathrm{ \,Mbyte/s}}}\xspace}
\def\mbsps      {\ensuremath{{\mathrm{ \,Mbytes/s}}}\xspace}
\def\gbsps      {\ensuremath{{\mathrm{ \,Gbits/s}}}\xspace}
\def\gbytes     {\ensuremath{{\mathrm{ \,Gbytes}}}\xspace}
\def\gbsps      {\ensuremath{{\mathrm{ \,Gbytes/s}}}\xspace}
\def\tbytes     {\ensuremath{{\mathrm{ \,Tbytes}}}\xspace}
\def\tbpy       {\ensuremath{{\mathrm{ \,Tbytes/yr}}}\xspace}

\def\dst        {DST\xspace}

%%%%%%%%%%%%%%%%%%%%%%%%%%%
% Detector related
%%%%%%%%%%%%%%%%%%%%%%%%%%%

%% Detector technologies
\def\nonn {\ensuremath{\mathrm{{ \mathit{n^+}} \mbox{-} on\mbox{-}{ \mathit{n}}}}\xspace}
\def\ponn {\ensuremath{\mathrm{{ \mathit{p^+}} \mbox{-} on\mbox{-}{ \mathit{n}}}}\xspace}
\def\nonp {\ensuremath{\mathrm{{ \mathit{n^+}} \mbox{-} on\mbox{-}{ \mathit{p}}}}\xspace}
\def\cvd  {CVD\xspace}
\def\mwpc {MWPC\xspace}
\def\gem  {GEM\xspace}

%% Detector components, electronics
\def\tell1  {TELL1\xspace}
\def\ukl1   {UKL1\xspace}
\def\beetle {Beetle\xspace}
\def\otis   {OTIS\xspace}
\def\croc   {CROC\xspace}
\def\carioca {CARIOCA\xspace}
\def\dialog {DIALOG\xspace}
\def\sync   {SYNC\xspace}
\def\cardiac {CARDIAC\xspace}
\def\gol    {GOL\xspace}
\def\vcsel  {VCSEL\xspace}
\def\ttc    {TTC\xspace}
\def\ttcrx  {TTCrx\xspace}
\def\hpd    {HPD\xspace}
\def\pmt    {PMT\xspace}
\def\specs  {SPECS\xspace}
\def\elmb   {ELMB\xspace}
\def\fpga   {FPGA\xspace}
\def\plc    {PLC\xspace}
\def\rasnik {RASNIK\xspace}
\def\elmb   {ELMB\xspace}
\def\can    {CAN\xspace}
\def\lvds   {LVDS\xspace}
\def\ntc    {NTC\xspace}
\def\adc    {ADC\xspace}
\def\led    {LED\xspace}
\def\ccd    {CCD\xspace}
\def\hv     {HV\xspace}
\def\lv     {LV\xspace}
\def\pvss   {PVSS\xspace}
\def\cmos   {CMOS\xspace}
\def\fifo   {FIFO\xspace}
\def\ccpc   {CCPC\xspace}

%% Chemical symbols
\def\cfourften     {\ensuremath{\mathrm{ C_4 F_{10}}}\xspace}
\def\cffour        {\ensuremath{\mathrm{ CF_4}}\xspace}
\def\cotwo         {\ensuremath{\mathrm{ CO_2}}\xspace} 
\def\csixffouteen  {\ensuremath{\mathrm{ C_6 F_{14}}}\xspace} 
\def\mgftwo     {\ensuremath{\mathrm{ Mg F_2}}\xspace} 
\def\siotwo     {\ensuremath{\mathrm{ SiO_2}}\xspace} 

%%%%%%%%%%%%%%%
% Special Text 
%%%%%%%%%%%%%%%
\newcommand{\eg}{\mbox{\itshape e.g.}\xspace}
\newcommand{\ie}{\mbox{\itshape i.e.}\xspace}
\newcommand{\etal}{\mbox{\itshape et al.}\xspace}
\newcommand{\etc}{\mbox{\itshape etc.}\xspace}
\newcommand{\cf}{\mbox{\itshape cf.}\xspace}
\newcommand{\ffp}{\mbox{\itshape ff.}\xspace}
\newcommand{\vs}{\mbox{\itshape vs.}\xspace}
 % Add in the predefined LHCb symbols
% Put here the symbols that are specific to this paper

% REDEFINE SOME GREEK LETTERS ####
\renewcommand{\epsilon}{\varepsilon}
\renewcommand{\theta}{\vartheta}

% FIXME
\newcommand{\fixme}[1]{{\color{red} FIXME: #1}}

% Others
\newcommand{\fsd}{\ensuremath{f_s/f_d}\xspace}
 % Add in user-defined symbols

% Automatic formatting of references
\usepackage[nameinlink,capitalise]{cleveref}
